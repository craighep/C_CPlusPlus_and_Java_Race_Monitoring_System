\documentclass[a4paper]{article}
\usepackage{graphicx} \usepackage{caption} \usepackage{pdfpages}
\usepackage{pdflscape} \usepackage[margin=0.8in]{geometry} \usepackage{fancyhdr}
\usepackage{pdfpages} 
\usepackage{longtable}
\usepackage{array}
\usepackage{lscape}
\begin{document}
\pagestyle{fancy}
\fancyfoot[R] {\thepage\ of 5}
\fancyfoot[L] {Crh13}
\fancyhead[LO]{CS22510 Assignment Two \hspace{5 mm}v1.2} %Name of doc, version,
Release
\fancyfoot[C]{Aberystwyth University / Computer Science} %Footer - no changes
necessary
\begin{center}
\textsc{\LARGE CS22510-  C++, C And Java Programming Paradigms }\\
\vspace{10 mm}
\textsc{\LARGE Language Choice and Comparisons of Assignment 1}\\[0.5cm] %Edit
for document title

\vspace{60 mm}
\begin{minipage}{0.8\textwidth}
\begin{flushleft} \large
\emph{Author:}\\
Craig Heptinstall \emph{(crh13)}\\
\vspace{10 mm}
Version 1.2, Release
\end{flushleft}
\end{minipage}
\vspace{80 mm}

%Computer Science Dept Address
\begin{minipage}{0.8\textwidth}
\begin{flushleft} \large
Department of Computer Science\\
Aberystwyth University\\
Aberystwyth\\
Ceredigion\\
SY23 3DB\\
\end{flushleft}
\end{minipage}
\vfill
{\large \today}
\end{center}
\clearpage
\setlength\parindent{0pt}

%Table of contents

\tableofcontents
\clearpage

\section{Introduction}
\subsection{Purpose of this Document}
This report provides my solution to assignment two of CS22510 (C, C++ and Java)
regarding the three programs I created in assignment one. I will attempt to
answer all the required specifics of the brief \cite{brief} given to me to the
best of my ability.
I will reference any pages or useful guides I came across in the references
section of this report. I would just like to note that the 2000 work word limit
has been kept here, and to ensure this document in within it, I used an on-line
Latex word count program \cite{words}.

\section{Language Choices}
I will now briefly discuss the language choices I made for each of the programs
created in the previous assignment of this module concerning the runners and
riders event creator, checkpoint manager and event manager systems.
Starting with the first of these, the event creator program required me to
create a system that could allow users to input the event name, entrants and
tracks. Because this program required only reading input from the user and then
writing this to the specified file, I felt that C++ would be the ideal language
to be used.
 \\[4mm]
 My reasoning for this was firstly the 'cin' function available within the
language. The cin was a great tool as it allowed for a wide range of inputs
under one function unlike having to identify the specific type that was being
read in, for instance a String, Int or any other type.
This meant I could easily read in a range of inputs for the required files that
would be written. This was similar for the output, as 'cout' easily allowed
output of text to the console.
To summarise this, the bitwise operators '\textgreater\textgreater' and
'\textless\textless' was at the heart of my choice here because of the ease of
input and output of a program where this was the key functionality.
 \\[4mm]
Looking at the next section of the runners and riders that also needed
implementing in one of the three languages was the checkpoint manager program.
For this, I chose to implement it with the Java programming language. 
When I was handed this assignment and found out that a graphical interface based
program would be a required part, I instantly decided that Java would be the
best option. My reasons for this language mainly lied upon the way I could
easily create the GUI's and transfer data about from user input to the forms.
Although the Gui was not the main part of this program, I still think that by
using Java it made it much easier to create the interface quickly allowing me to
concentrate on the use of reading and writing files to the users desire. 
I would also like to add that because Java is a portable language, and the
system here would be intended for being carried around a track when updating
checkpoints, it was an ideal solution if it was on a range of systems.
 \\[4mm]
The final program in the first assignment was the one I used the C language to
create. This was the program for the general manager of race events allowing
several options for the user, all in command line. 
Because C is a powerful language, and this program was to be a command line
system, this is the start of my reason to choose it for such a program. Firstly,
the simplicity of moving around and checking data was ideal for the wide range
of functions that are implemented. 
For instance, using Structures in C was a easy way to store sets of data for an
entrant. The main reason I chose C was the strictness that the language brings.
It meant that if any data did not relate to the header file available, or
methods were implemented that 
should be, the program would report an error. This was very important to me
because this part of assignment one was the most important because of the wide
variety of options in the system.
 
\section{Comparisons of Languages Chosen}	
For the next section of the report, I have been asked to compare and contrast
these different languages taking into account my choices for which languages I
chose to implement the different sections of the system with. 
Therefore, I will be firstly addressing the use of input and output for each of
the languages. Starting with C, this was the language I found that although
could read in the same types as C++ and Java, it required more code and work.
This was because I had to specify what type was being read in for instance a
float which was \%f rather than a \%d for integers. Java works on a similar
principle although can handle floats and integers being read in using the same
method. 
Meanwhile C++ can handle a wider range of input types under the same cin method
. I found that although this meant it was easier to code, C was the best
performing here because of the stricter method. It meant that any wrong types
could not be inserted into the wrong format variables.
 \\[4mm]
As for the outputting methods of each of the languages, this was the similar
story to the inputs of each. When I was outputting in Java, this was the
simplest of the three, by using using the 'println' statement. In C++, similar
to Java in the sense of outputting any types into the console in one line,
although I did not know how to output multiple variables at once. After looking
this up I found using the binary operators '\textless\textless' was required
after each variable or text. Hence Java being the best in comparison for
outputting onto console.
 \\[4mm]
The next set of comparisons I want to look at is the way I stored data with each
of the three languages. For instance, Java offers an alternative to the other
two, by using Objects rather than structures \cite{differences}. In C and C++ 
parts of the first assignment I found using structures somewhat harder to use 
at first, being because that structures cannot have methods inside each structure 
while Java could. Because of this, it emphasised my reason for the Java choice 
for the checkpoint manager. This leads me onto the next major difference concerning 
the storage of multiple entrants for instance in an array. While Java allowed me to 
create an ArrayList, meaning I could always add to it easily, C and C++ required t
he use of malloc and calloc. This made Java a lot easier to use when juggling 
around arrays of entrants, while in C and C++ for simplicity I used set sized 
arrays and made them to a size I would find suitable for any data inputted for an
event. I should also mention that pointers are an item of the C and C++
languages 
that Java does not have an equivalent to. This is because Java passes data by
value rather than having the option to pass  by reference unlike the other two.
This meant easier and wider options of moving data around my programs for the
event creation and event managers.
 \\[4mm]
The final two compare and contrasts concern the strictness of the languages and
also the graphical capabilities for each language. Especially in C++, I found
that this language uses very strict type checking. For instance, a prototype
must be known for each function that is called, and the call must match that
prototype. In Java the language is much more flexible, and C similarly that does
not have such checking but does allow header files which ensure types are
implemented and signified.
 \\[4mm]
 Looking at the graphical user interface implementation possibilities of the
three languages, because I have had a good deal of experience in Java, this
swing my decision to create the checkpoint manager in it to save time during the
project. I must also mention that from looking at tutorials and discussions
concerning GUI's in C and C++ \cite{gui}. In my experience of trying the three,
Java was the easiest because of its built in libraries for the creation of GUI's
meanwhile C and C++ required additional libraries such as QT. The portability of 
Java is usually considered much more portable than both C and C++ because of the 
built libraries. Between C and C++, C is more portable than its counterpart because 
of the need for less libraries. \cite{portable}
 \\[4mm]
As a last point, regarding the code layout and comments of each program, Java
has the Java Doc which provides in line commenting of methods and variables
automatically whilst C and C++ does not provide this without additional support.
Java files are all usually placed in packages as I used in my program, yet the
other two have their files usually all together and to call another method the
program must import the required classes. Although in Java these classes
represent as objects. Generally, the speed is also faster from looking at reviews 
and general information pages \cite{speed} than Java at compile/ run time when it 
comes to the two other languages. 

\section{Discussion of Alternative Options}
Now I have completed both the first section regarding the choices I made
considering each of the three languages the entire system was created in, and
compared each of these in terms of implementing code, I will now consider any
possible changes I would make if I were to do a similar project again. For 
instance, choosing to use different languages for the programs in a different 
order to what I chose this time around. 
 \\[4mm]
The first change I would make looking at the event manager program. In the 
previous section, I said C++ was more strict than the other two languages, 
therefore I think if the manager program of the event was written in this 
language rather than C would mean it would be less error prone, and incorrect
data would produce errors. Using C++ for the event manager would be ideal 
because this program was the one that handled more data and options to the users. 
order to what I chose this time around. 
 \\[4mm]
The next change I would have liked to implement if I did this project a second 
time around would be changing the program used to construct the GUI with for 
the checkpoint manager. Rather than using Java, although easier with my current 
knowledge it would be to construct, I think after learning C graphical programming 
I would be able to construct similar if not better GUI's. Firstly I noted previously 
that there are a range of libraries available to construct graphical programs 
with C available. This would give me more of an option in my choice of elements 
contained in the GUI. I also noted that speed was a difference between C and Java 
with C being faster. This would mean that the graphical programs would load 
faster for users which would provide especially useful for people 
quickly attempting to register a runner/ rider on the fly. The only issue I 
would have to overcome with this change would be that It could prove more 
of a problem concerning portability using C over Java for such a program 
with the extra required libraries.
 \\[4mm]
Finally, With the first program from assignment one, which would be the last remaining 
out of my choices for a language, Java would be the one to program this in. Apart 
from choosing this as the last language and program remaining, there are still 
reasons to choose this language appropriate for this type of system. For instance, 
rather unlike in C where I had to have structures string data in the same C file, 
Java would allow me to easily separate data being inputted into separate classes 
essentially turning them into objects. This would organise the data better and keep
objects separated clearer.

\vspace{10mm} 
\begin{center}
\framebox[1.4\width]{Word Count: 1966}
\end{center}

\vspace{4mm}
\section{References}

\begin{thebibliography}{}

\bibitem{brief} CS22510 Assignment 2 Brief. Neal Snooke, Fred Labrosse.
2012-2013.
\bibitem{words} TexCount Web Service. V2.3.
http://app.uio.no/ifi/texcount/contact.html.
\bibitem{differences} Differences in Java and C/C++, including data storage.
http://www.programmersheaven.com/2/FAQ-JAVA-Differences-Between-JAVA-And-C-CPP. 2010.
\bibitem{gui} Stack Over Flow. http://goo.gl/eY4uq.
\bibitem{portable} DI Management Site. Why C is better than C++. 
http://www.di-mgt.com.au/cprog.html. 2010
\bibitem{speed} Prof. Cliff Click Blog. http://www.azulsystems.com/blog/cliff/2009-09-06-java-vs-c-performanceagain. 2009

\end{thebibliography}

\section{Document History}

\begin{tabular}{|l | l | l | l | l |}

\hline

Version & Date & Changes made to Document & Changed by \\

\hline

1.0  & April 7, 2013 & Creation of first draft of document & Crh13\\
\hline
1.1  & April 11, 2013 & Amendments and spell checking of document & Crh13\\
\hline
1.2  & \today & Release & Crh13\\
\hline


\end{tabular}
\end{document}

\clearpage

\end{document}